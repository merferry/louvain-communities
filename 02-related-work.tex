The \textit{Louvain} method is a greedy modularity-optimization based community detection algorithm \cite{com-blondel08}. It identifies communities with resulting high modularity, and is thus widely favored \cite{com-lancichinetti09}. Algorithmic improvements proposed for the original algorithm include early pruning of non-promising candidates (leaf vertices) \cite{com-ryu16, com-halappanavar17, com-zhang21, com-you22}, attempting local move only on likely vertices \cite{com-ryu16, com-ozaki16, com-zhang21, com-shi21}, ordering of vertices based on node importance \cite{com-aldabobi22}, moving nodes to a random neighbor community \cite{com-traag15}, threshold scaling \cite{com-lu15, com-naim17, com-halappanavar17}, threshold cycling \cite{com-ghosh18}, subnetwork refinement \cite{com-waltman13, com-traag19}, multilevel refinement \cite{com-rotta11, com-gach14, com-shi21}, and early termination \cite{com-ghosh18}.

To parallelize the Louvain algorithm, a number of strategies have been attempted. These include parallelizing the costly first iteration \cite{com-wickramaarachchi14}, performing iterations asynchronously \cite{com-que15, com-shi21}, ordering vertices via graph coloring \cite{com-halappanavar17}, using vector based hashtables \cite{com-halappanavar17}, using adaptive parallel thread assignment \cite{com-fazlali17, com-naim17, com-mohammadi20}, and using sort-reduce instead of hashing \cite{com-cheong13}\ignore{, using simple partitions based of vertex ids \cite{com-cheong13, com-ghosh18}, and identifying and moving ghost/doubtful vertices \cite{com-zeng15, com-que15, com-bhowmik19, com-bhowmick22}}.

The \textit{Label Propagation Algorithm (LPA)} is a diffusion-based technique for identifying communities. It is faster and more scalable than Louvain, as it does not require repeated optimization steps and is easier to parallelize \cite{com-newman04, com-raghavan07}. Improvements upon the LPA include using a stable (non-random) mechanism of label choosing in the case of multiple best labels \cite{com-xing14}, addressing the issue of monster communities \cite{com-berahmand18, com-sattari18}, identifying central nodes and combining communities for improved modularity \cite{com-you20}, and using frontiers with alternating push-pull to reduce the number of edges visited and improve solution quality \cite{com-liu20}.

A few open source implementations and software packages have been developed for community detection. Vite \cite{ghosh2018scalable} is a distributed memory parallel implementation of the Louvain method that incorporates several heuristics to enhance performance while maintaining solution quality, while Grappolo \cite{com-halappanavar17} is a shared memory parallel implementation. Fast Label Propagation Algorithm (FLPA) \cite{traag2023large} is a fast variant of the LPA, which utilizes a queue-based approach to process only vertices with recently updated neighborhoods. NetworKit \cite{staudt2016networkit} is a software package designed for analyzing the structural aspects of graph data sets with billions of connections. It is implemented as a hybrid with C++ kernels and a Python frontend, and includes parallel implementations of Louvain and LPA. igraph \cite{csardi2006igraph} is a similar package, written in C, with Python, R, and Mathematica frontends. It is widely used in academic research, and includes implementations of Louvain and LPA.
