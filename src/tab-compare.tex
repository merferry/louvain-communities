\begin{table}[hbtp]
  \centering
  \caption{Speedup of our multicore implementation of Louvain algorithm compared to other state-of-the-art implementations. Direct comparisons are based on running the given implementation on our server, while indirect comparisons (denoted with a $*$, details given in Section \ref{sec:comparison-indirect}) involve comparing results obtained by the given implementation relative to a common reference (Grappolo). Among the Louvain implementations, some are multi-core, while others are categorized as multi-node (\textit{nodes}), hybrid GPU (\textit{GPU}), multi-GPU (\textit{GPUs, 1 GPU}), and multi-node multi-GPU (\textit{DGPUs}).}
  \label{tab:compare}
  \begin{tabular}{|c|c||c|}
    \toprule
    \textbf{Louvain implementation} &
    \textbf{Published} &
    \textbf{Our Speedup} \\
    \midrule
    Grappolo \cite{com-halappanavar17} & 2017 & $22\times$ \\ \hline
    Vite \cite{ghosh2018scalable} & 2018 & $50\times$ \\ \hline
    NetworKit Louvain \cite{staudt2016networkit} & 2016 & $20\times$ \\ \hline
    cuGraph Louvain \cite{kang2023cugraph} & 2023 & $5.8\times$ \\ \hline
    % Qie et al. \cite{qie2022isolate} & 2022 & $273\times^*$ \\ \hline
    PLM \cite{staudt2015engineering} & 2015 & $48\times^*$ \\ \hline
    APLM \cite{com-fazlali17} & 2017 & $7.7\times^*$ \\ \hline
    % DPLAL (nodes) \cite{sattar2022scalable} & 2022 & $5472\times^*$ \\ \hline
    Ghosh et al. (\ignore{8 }nodes) \cite{com-ghosh18} & 2018 & $5.9\times^*$ \\ \hline
    HyDetect (GPU) \cite{com-bhowmik19} & 2019 & $54\times^*$ \\ \hline
    Rundemanen (GPU) \cite{com-naim17} & 2017 & $9.8\times^*$ \\ \hline
    ACLM (GPU) \cite{com-mohammadi20} & 2020 & $6.0\times^*$ \\ \hline
    Nido (1 GPU) \cite{chou2022batched} & 2022 & $9.2\times^*$ \\ \hline
    cuVite (1 GPU) \cite{com-gawande22} & 2022 & $6.7\times^*$ \\ \hline
    Cheong et al. (\ignore{16/24 }GPUs) \cite{com-cheong13} & 2013 & $8.3\times^*$ \\ \hline
    Bhowmick et al. (\ignore{8 }DGPUs) \cite{com-bhowmick22} & 2022 & $4.9\times^*$ \\ \hline
  \bottomrule
  \end{tabular}
\end{table}
